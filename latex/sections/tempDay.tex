\begin{document}

\begin{wrapfigure}{r}{0.5\textwidth}
\centering
\includegraphics[scale=0.6]{flowcharttempday.png}
\caption{\label{fig:flowcharttempday} Flow chart for the function that gives the temperature of a given day.}
\end{wrapfigure}

Knowing the temperature of a given day does not confirm global warming by itself, but it can be interesting with more detailed data on certain days for other reasons. For example, is the weather the same on midsummer and christmas? 

Two functions were created, both giving the temperature of a certain day. The difference is that one takes the date (month, day) as input and the other one takes the day number as input. To make it as simple as possible, the function that takes day number as input first coverts it into month and day, so that the functions thereafter are identical. 

The algorithm itself is straightforward. First, the dates in the data that matches the input date are found. Then, the temperatures for all years are stored in a vector, which is plotted in a histogram. This process is shown in a flow chart in Fig. \ref{fig:flowcharttempday}. 

\begin{figure}[ht]
\begin{center}
\includegraphics[scale=0.45]{tempOnDayNumber.png}
\caption{\label{fig:tempOnDayNumber} The temperatures on July 19$^{\text{th}}$ in Uppsala for the years between 1722 and 2013. The black line is a Gaussian fitted to the histogram.}
\end{center}
\end{figure}

A histogram for the 19$^{\text{th}}$ of July in Uppsala, for the years from 1722 to 2013, is shown in Fig. \ref{fig:tempOnDayNumber}. From the histogram it is possible to get both the mean and the standard deviation. For Fig. \ref{fig:tempOnDayNumber}, the mean is 16.88 and the standard deviation is 2.96. If we want to know the probability for a certain temperature on the given day, we can use the mean and the standard deviation and assume a Gaussian distribution. The black line in Fig. \ref{fig:tempOnDayNumber} is a Gaussian fitted to the histogram, to see if it is a reasonable assumption. For Uppsala, the data contains enough years to give 274 counts, so the Gaussian fit should be sensible. This is not true, given that the fit has a $\chi ^2$ of 2, which is not good at all. Some of the other data sets would give a lot less data points, as in Fig. \ref{fig:tempOnDay}. The histogram contains only 35 entries for the 19$^{\text{th}}$ of July in Lund, and the Gaussian fit is definitely not a good approximation. Therefore, the conclusion is that it is not possible to calculate the probability of a certain temperature by approximating with a Gaussian.

\begin{figure}[ht]
\begin{center}
\includegraphics[scale=0.45]{tempOnDay.png}
\caption{\label{fig:tempOnDay} The temperatures on July 19$^{\text{th}}$ in Lund between 1961 and 2014. The black line is a Gaussian fitted to the data.}
\end{center}
\end{figure}


Midsummer is celebrated in Sweden on a day between June 19$^{\text{th}}$ and June 26$^{\text{th}}$. Historically, beginning in the 4th century, Midsummer was celebrated on the 24$^{\text{th}}$ of June. Therefore, June 23$^{\text{rd}}$ is chosen for the comparison since we celebrate on midsummers eve (we now celebrate on the Friday between June 19$^{\text{th}}$ and June 26$^{\text{th}}$). The weather on midsummers eve is compared to the weather on christmas eve, both using the dataset from Uppsala, seen in Fig. \ref{fig:midsummerchristmas}. Clearly there is a large difference in the temperatures on midsummer and christmas eve. The mean for midsummers eve (Fig. \ref{fig:midsummerchristmas}(a)) is 15.4 $^{\circ}$C and the mean for christmas eve (Fig. \ref{fig:midsummerchristmas}(b)) is -3.1 $^{\circ}$C.

\begin{figure}[ht]
\centering
\subfloat[Midsummers eve]{\label{fig:midsummer}\includegraphics[width=0.5\textwidth]{tempmidsummer.png}} 
\subfloat[Christmas eve]{\label{fig:christmas}\includegraphics[width=0.5\textwidth]{tempchristmas.png}}\\
\caption{The temperatures on midsummers eve (a) and christmas eve (b), in Uppsala, for the years between 1722 and 2013.}
\label{fig:midsummerchristmas}
\end{figure}



\end{document}